% Created 2019-06-05 Wed 10:52
% Intended LaTeX compiler: pdflatex
\documentclass[11pt]{article}
\usepackage[utf8]{inputenc}
\usepackage[T1]{fontenc}
\usepackage{graphicx}
\usepackage{grffile}
\usepackage{longtable}
\usepackage{wrapfig}
\usepackage{rotating}
\usepackage[normalem]{ulem}
\usepackage{amsmath}
\usepackage{textcomp}
\usepackage{amssymb}
\usepackage{capt-of}
\usepackage{hyperref}
\author{cat}
\date{\today}
\title{}
\hypersetup{
 pdfauthor={cat},
 pdftitle={},
 pdfkeywords={},
 pdfsubject={},
 pdfcreator={Emacs 26.1 (Org mode 9.1.9)}, 
 pdflang={English}}
\begin{document}

\tableofcontents

\section{What I did today}
\label{sec:org49ad699}
\begin{enumerate}
\item Use Haskell package to simplify Html multiple lines in Haskell\textit{<2019-05-09 Thu>}
\begin{verbatim}
package Text.RawString.QQ

multipline = [r| 
	     <p>"dog"</p>
	     <p>"cat"</p>
	     |]
\end{verbatim}

\item Added code to process background process in Shell from Web Input in wai server\textit{<2019-05-10 Fri>}
\begin{verbatim}
"x nohup filewatcher &" 
\end{verbatim}

\item ghci does not load package into Ghci if you use :load myhaskell.hs
It causes lots of confusing when you load your code inside ghci
\end{enumerate}

\section{Added Javascript to edit code block}
\label{sec:org9a103a4}
\begin{enumerate}
\item Added Javascript to listen to onclick event, e.g. enable/disable TextArea visibility
\item Hide header to TextArea with css \texttt{display:none}
\item Generate id for each code block in TextArea
\item Go through each code block and remove the edited code block from a list
\item Append the edited the code block to the new list
\end{enumerate}

\section{MacOS automatically update screws up \textbf{tmux}}
\label{sec:org177e190}
\begin{enumerate}
\item \textbf{tmux} can not start up
\begin{itemize}
\item reinstall tmux:
\begin{verbatim}
brew uninstall --force tmux
brew install tmux
\end{verbatim}
\end{itemize}
\end{enumerate}



\textbf{`'}  ` `dd vim script to check definition of Haskell function with Timer and \texttt{job\_start()}
\begin{enumerate}
\item Create a timer which calls a function every n seconds, e.g. 0.5 second \texttt{timer\_start()}
\item The function contains code to start a job or process in background and query a value from key-value \textbf{Redis server} \texttt{redis\_query.hs}
The key is the current \textbf{word} under current \textbf{cursor}
\item The function calls \texttt{systemlist()} and passes \texttt{\$b/redis\_query.hs} as an argument and gets the \textbf{return list} which contains the values from \textbf{Redis server}
\item Use \texttt{echo} to display the \textbf{return list} at the bottom of \textbf{Vim}
\end{enumerate}

\section{Haskell show, print and putStr or putStrLn}
\label{sec:orgdfb1d1a}
\begin{enumerate}
\item show, print and putStr \texttt{"dog\textbackslash{}\textbackslash{}"},
\begin{verbatim}
show "dog\\" -- "\"dog\\\\\""
print "dog\\" -- "dog\\"
putStr "dog\\" -- dog\
\end{verbatim}
\end{enumerate}
\section{Add wai-uil to wai.hs Server for redirecting response URI}
\label{sec:orgabd8723}
\begin{enumerate}
\item Install wai-uti with \texttt{cabal install wai-uti}
\item Using \texttt{response <<= redirect' status302 [] uri where Just uri = parseURI "http://newuri.com"}
\href{https://bitbucket.org/zsurface/haskell\_webapp/src/master/wai.hs}{wai.hs}
\end{enumerate}

\section{Solve "Insert a node into a binary tree", it seems to be an easy question, but it is tricky to get it right on the first try.}
\label{sec:org7cb1fd7}
\begin{itemize}
\item Two ways to solve the problem.
\begin{enumerate}
\item Use \textbf{Iteration} with \texttt{while}
\begin{itemize}
\item Start from the root node as current node.
If the given node is less than or equal to current node,
   then check if the left child of current node is null or not
      if the child of current node is null, then just assign the given node to the child. Done
      else using \texttt{tmp = tmp.left} to goto next child of current node.
otherwise repeat the same thing in the right subtree.
\end{itemize}
\item Use \textbf{Recursion}, it is almost the same as Iteration as above, but the root needs to be outside of the method.
\end{enumerate}
\item Use recusion is \textbf{tricky} because if the root node is null, then root node can not be modified inside a method.
\item The best way to do is to use \textbf{Iteration}
\begin{verbatim}
   // Iteration
public static void insert(Node node){
    Node curr = root;
    // left subtree
    if(node.data <= curr.data){
	if(curr.left == null)
	    curr.left = node;
	else
	    curr = curr.left;
    }else{
	// right subtree
	if(curr.right == null)
	    curr.right = node;
	else
	    curr = curr.right;
    }
}

   // Recursion
public static void insert(Node root, Node node){
    if(root != null){
	if(node.data <= root.data){
	    if(root.left == null)
		root.left = node;
	    else
		insert(root.left, node);
	}else{
	    if(root.right == null)
		root.right = node;
	    else
		insert(root.right, node);
	}
    }
}
\end{verbatim}
\end{itemize}

\subsection{{\bfseries\sffamily DONE} hair cut}
\label{sec:orgbdb5a83}
\begin{enumerate}
\item 17\$ for the stupid hair cut.
\end{enumerate}
\subsection{{\bfseries\sffamily DONE} fix add, update and delete buttons on wai.hs}
\label{sec:orgbb4b3eb}
\begin{enumerate}
\item There is some issue in with runSh "mv" to rename file, after remove "mv" it seems to be working
\end{enumerate}
\subsection{{\bfseries\sffamily DONE} Do my laundry}
\label{sec:org5d9f74e}
\subsection{{\bfseries\sffamily DONE} Install Vim 8.1.135, and fix symbol link with the Vim syntax.vim file error}
\label{sec:org4a38c54}
\subsection{{\bfseries\sffamily DONE} Try to understand Visitor pattern in Java and Haskell Algebraic Datatypes}
\label{sec:org13e06fa}
\subsection{{\bfseries\sffamily DONE} Add Redirect response to wai.hs with wai-uti}
\label{sec:org3a5daaf}
\subsection{{\bfseries\sffamily DONE} pushall respositories to bitbucket}
\label{sec:orgdb94095}
\subsection{{\bfseries\sffamily DONE} fix resume a bit}
\label{sec:org6e3e9d3}
\subsection{{\bfseries\sffamily TODO} Need to add more detail to resume}
\label{sec:org907efcc}
\subsection{{\bfseries\sffamily TODO} Buy a new pot and buy some rice}
\label{sec:orgb15fa8d}
\subsection{{\bfseries\sffamily DONE} Open new paypal account}
\label{sec:org8b3a095}
\subsection{{\bfseries\sffamily DONE} Add color to wai.hs to highlight the current edited code block}
\label{sec:orgeb939ac}
\begin{enumerate}
\item when current code block is clicked, change the background of pre
\item Need a different pre for the current code block because all the pre use the same style.
\item Add new class name, and id to pre tags and rename the class name when the code block is clicked.
\end{enumerate}

\subsection{{\bfseries\sffamily DONE} Send out more resumes}
\label{sec:orga4e0562}
\subsection{{\bfseries\sffamily TODO} Do some java coding}
\label{sec:org3bed8b8}
\begin{itemize}
\item Rotate square array to left/right 90 degrees.
\item Balance brackets.
\item Rotate spiral rectangle.
\item Insert a node into a binary tree.
\item Find all the nodes that are distance k from a given node
\item Check whether a node is in the binary tree or not
\item Single linked, append, addFront and remove operations, it seems to be working
\item Added html file filter to genehtml.hs to avoid error when genehtml reads non-html file.
\item Implement insect operation for HashMap with array and single linkedlist.
\begin{enumerate}
\item Remember to increase the count if a node is inserted
\item If hash key collision occurs, then the value needs to be checked whether it is in the list
\item When walking through the list, previous node need to be kept.
\item Before the insection, count need to be check whether it less than the \textbf{max} size.
\end{enumerate}
\item Implement \textbf{HashMap} insect operation using binary tree.
\item Check a binary tree is whether a \textbf{BST}.
\begin{enumerate}
\item Use backtrack.
\item Use \textbf{BST} definition.
\end{enumerate}
\item Added PriorityQueue code example, \texttt{java.util.PriorityQueue}, and use lambda for comparatoin,.
Add code using \texttt{Comparable<Person>} and \texttt{Comparator<Person>}
\begin{verbatim}
class Person{
   String firstName;
   String lastName;
   public Person(String f, String l){
      this.firstName = f;
      this.lastName = l;
   }
}

// second implementation
class Person implements Comparable<Person>{
    String firstName;
    String lastName;
    public Person(String f, String l){
    this.firstName = f;
    this.lastName = l; 
   }
   public int compareTo(Person other){
     return this.firstName.compareTo(other.firstName);
   }
}

PriorityQueue queue = new PriorityQueue((\a, b) -> a.firstName.compareTo(b.firstName));
queue.add(new Person("David", "lee"));
queue.add(new Person("Tommy", "kee"));
queue.add(new Person("Jacky", "kuu"));
while(!queue.isEmpty()){
   Person p = queue.remove();
   System.out.print(p.firstName);
}

// use Comparator interface,                                
class PersonCmp implements Comparator<Person>{             
     public int compare(Person p1, Person p2){             
       return p1.firstName.compareTo(p2.firstName);        
     }                                                     
}                                                          
Collections.sort(new PersonCmp());                         
\end{verbatim}
\item Added merge two sorted lists.
\item Added Lease Recent Used code.
\item Added Double linked List, delete, append and insertFront.
\item Read file line by line and split them.
\item Serialize and deserialize binary tree with map.
\begin{itemize}
\item Use map with \(k=0, 2*k + 1, 2*k + 2, \dots\) to index binary node.
\item File, BufferedReader, BufferedWriter, \texttt{BufferedReader br = new BufferedReader(new FileReader(fname))}
\item Use \textbf{preorder} traveral to write all the keys and nodes to file.
\begin{verbatim}
public void serialize(Node curr, Integer k, BufferedWriter bw){
    // preorder traveral
    // handle IOException here
    // k=0, k=1, k=2, k=3 ...
    bw.write(k + " " + curr.data + "\n");
    serialize(curr.left, 2*k + 1);
    serialize(curr.right, 2*k + 2);
}

public Node deserialize(Integer k, Map<Integer, Integer> map){
    // handle IOException here
    Integer v = map.get(k);
    if(v != null){
	Node root = new Node(v);
	root.left = deserialize(2*k + 1, map);
	root.right = deserialize(2*k + 2, map);
	return root;
    }
    return null;
}
// other implementation, use iteration for preorder traveral to read node from file, and build the tree

\end{verbatim}
\item Use \textbf{preorder} traveral from map and build a tree \textbf{from bottom and up}.
\end{itemize}
\end{itemize}


\subsection{{\bfseries\sffamily TODO} Take a look at the Algebraic Data Type and Visitor pattern in Java}
\label{sec:orgff86d82}
\href{https://www.ahnfelt.net/monads-forget-about-bind/}{URL}
\subsection{{\bfseries\sffamily DONE} Send out more resume,  more to EA}
\label{sec:org56aba22}





\subsection{{\bfseries\sffamily DONE} Add Applescript to send notification when filewatcher detects some txt file inside password folder.}
\label{sec:org25a4dff}
\subsection{{\bfseries\sffamily DONE} Finally figure out how to use multiple putStrLn in Agda}
\label{sec:org5621462}

\subsection{{\bfseries\sffamily DONE} Refactor filewatcher.hs, remove IORef, clean up useless code}
\label{sec:org15b4584}
\subsection{{\bfseries\sffamily DONE} Added double click: \texttt{ondblclick} event to javascript file inside wai.hs to improve user experience.}
\label{sec:org84a45ef}
\subsection{{\bfseries\sffamily DONE} Finished osense take home assignment}
\label{sec:org8151c1b}
\begin{enumerate}
\item Use C standard library to display statistic temperature info from a CVS file.
\begin{itemize}
\item Read the CVS file line by line
\item Allocate memory to store all the temperatures in an array.
\item Write a quick sort function to sort the array
\item From the sorted array, mean, median, mode, min, max and standard deviation can be compuated.
\end{itemize}
\end{enumerate}





\subsection{{\bfseries\sffamily DONE} Added gcc compile script to vimrc file, use \texttt{\$scr/gcc\_compile.sh}}
\label{sec:org64be33c}

\subsection{I work on Application generate report for business department}
\label{sec:org6dd1615}
\begin{itemize}
\item The Applicaiton take data from database such as sale information,
\item and we do accounting calculation based on the data.
\item We generate report at the end of each months
\item The report store in S3 which Amazon file storage,
\item The business department can download the report from S3.
\item Technologies that we used:
\item The application is Java based.
\item We use:
\item Hiberate use database modeling
\item Spring an Application framework, dependency injection, annotation, MVC Model View Controller
\end{itemize}
\subsection{In VoiceBox Inc.}
\label{sec:orgc551150}
\begin{itemize}
\item Development Application to generate nature language.
\item e.g. Given a set of rules, The App generate English language based on them.
\item e.g. What is the temperature in Vancouver Today,
\item Vancouver is a city, and
\item The set of rules is like "What is the temperature in City today"
\item What is the intentation of user,
\item e.g. do I need a jacket today?
\end{itemize}
\end{document}
