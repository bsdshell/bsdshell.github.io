% Created 2019-05-20 Mon 13:06
% Intended LaTeX compiler: pdflatex
\documentclass[11pt]{article}
\usepackage[utf8]{inputenc}
\usepackage[T1]{fontenc}
\usepackage{graphicx}
\usepackage{grffile}
\usepackage{longtable}
\usepackage{wrapfig}
\usepackage{rotating}
\usepackage[normalem]{ulem}
\usepackage{amsmath}
\usepackage{textcomp}
\usepackage{amssymb}
\usepackage{capt-of}
\usepackage{hyperref}
\author{cat}
\date{\today}
\title{}
\hypersetup{
 pdfauthor={cat},
 pdftitle={},
 pdfkeywords={},
 pdfsubject={},
 pdfcreator={Emacs 27.0.50 (Org mode 9.1.9)}, 
 pdflang={English}}
\begin{document}

\tableofcontents

\subsection{Monad}
\label{sec:org037e185}
\begin{enumerate}
\item Monad is monoid over the \textbf{endofunctor}
\item endofunctor is just domain and image are both same functor
\end{enumerate}
\subsection{definition of Monad}
\label{sec:orga7dd9ca}
η :: I → T  \\
μ :: T × T → T
\begin{align*}
 \mu &: T \times T \rightarrow T  \quad \text{ where } T \text{ is endofunctor}     \\ 
 \mu T &: (T \times T) \times T \rightarrow T^2  \\
 T \mu &: T \times (T \times T) \rightarrow T^2  \quad \text{Associativity law in Monoid}\\
 \mu T &= T \mu  \quad \text{from commutative diagram} \\
 T \mu \mu   &= T \\
 \mu T \mu &= T \\
 T \mu \mu &= \mu T \mu \\
 \eta &: I  \rightarrow T              \\ 
 \mu_a &: T \times T a \rightarrow T a \\ 
 \eta_a &: I a \rightarrow T a     \quad \text{ where } I \text{ identity endofunctor }    \\ 
\end{align*}
\begin{verbatim}
f(x) = x² + x³
A = B → B ∘ μ η
C  E   F = f(9) = 4
\end{verbatim}
Matrix
\[
    A= \begin{bmatrix}
    \cos(\beta) & -\sin(\beta)\\
    \sin(\beta) & \cos(\beta)
    \end{bmatrix} + 
    B = \begin{bmatrix}
       \cos \beta & -\sin \beta \\
       \sin \beta &  \cos \beta
       \end{bmatrix}
    \]
 Matrix multiplication
\begin{enumerate}
\item Matrix addition
\begin{enumerate}
\item matrix division
\end{enumerate}
\end{enumerate}

\section{Top Level Header}
\label{sec:org95b5c20}
\textbf{* Second Level header
*} second Level 2
** second Level 3
\begin{itemize}
\item this is what
\item this is cool
\item this is cool also
\item this is nice
\item this ia also good
\item nasdfkasdkfjaskdfj asdf
\item what the fuck
\item askdfj
\item ksjfkaskdf
\item iajsdk
\item askdfjaksd
\item aksdjfaksj
\item askdfjaksdfdf
\item asdfkasdf
\item nice
\item cool
\end{itemize}
\begin{center}
\begin{tabular}{lll}
name & phone & email\\
dog & cat & rat\\
→ & ⇒ & ⇐\\
β & η & μ\\
ω & β & α\\
δ & Σ & π\\
Ω & η & ζ\\
\end{tabular}
\end{center}

\subsection{My Alphebat}
\label{sec:org0ddcaeb}
\begin{center}
\begin{tabular}{llll}
a & b & c & d\\
\end{tabular}
\end{center}
\subsection{how to solve quadratic equation}
\label{sec:orgcb779aa}
\begin{enumerate}
\item how to create \(f(x) = x² + x³ + 2\)
\item How to find the root of quadratic equation?
\begin{enumerate}
\item How to create squares of the equation?
\item How to solve the equation?
\item How to use the quadratic equation?
\item How to find the root of quadratic quation?
\item How to the creat the quadratic equation?
\item How to find the root
\item how to
\end{enumerate}
\item hwo to asdf asdkfja sdf
\item how to wha tthat
\item how to find the solution of equation?
\item How to solve the java problem?
\item How to optimization of the problem?
\item How to move the cursor around
\item asdjfaksdjfkasjdkfjasdjf
Polynomial Eqation:
\item \(\sin α + \cos α = \sin π + \cos π\)
\end{enumerate}
\begin{verbatim}
f::(Monad m)=> m a -> (a -> m b) -> m b

transpose::[[a]] -> [[a]]
transpose [] -> repeat []
transpose (x:cs) = zipWith(:) x $ transpose cs
\end{verbatim}

\subsection{What is Functor}
\label{sec:orgf2ca174}
\begin{enumerate}
\item The definition of \textbf{Functor} in Haskell.
Functor is the type class with two methods,
\begin{verbatim}
class Functor f where
fmap::(a -> b) -> f a -> f b
\end{verbatim}
The instance of \textbf{Functor} has to satisfy following two laws:
\begin{verbatim}
fmap id = id
fmap (f . g) = (fmap f) . (fmap g)
\end{verbatim}
\begin{enumerate}
\item What is the difference between Functor and Monad
Monad is the subtype of Functor
\end{enumerate}
\end{enumerate}

\subsection{What is Monad}
\label{sec:org1225608}
\begin{enumerate}
\item What is the definition of Monad?
Monad is \textbf{Monoid} over the \textbf{endofunctor}
\begin{verbatim}
1  class Applicative m => Monad m where
2    return:: a -> m a
3    return = pure
4    (>>=)::(Monad m) => m a -> (a -> m b) -> m b
\end{verbatim}
\item When to use Monad?
\end{enumerate}

\subsection{What is Monoid and Monad}
\label{sec:org8fa69da}
\begin{enumerate}
\item What is the difference between Monad and Monoid?
There are couple \textbf{axioms} for \textbf{Monoid}
\begin{enumerate}
\item id ⊗ m = m ⊗ id = m
\item m1 ⊗ m2 ⊗ m3 = m1 ⊗ (m2 ⊗ m3)
\end{enumerate}
\item The mathematic definition of \textbf{Monad}   
\begin{enumerate}
\item μ :: I → T
\item η :: T ⊗ T → T
\item T is the endofunctor which means from a category to itself. (T : C → C)
\end{enumerate}
\item The domain and co-domain of η are both \textbf{Functor}
\begin{itemize}
\item ⊗ is \textbf{Functor} composition
\item e.g. \texttt{if T = m a then T ⊗ T = m (m a)}
\item e.g. \texttt{T = [] then T ⊗ T = [[]]}
\item In Haskell, type constructor is like a \textbf{Functor}
\end{itemize}
\begin{verbatim}
class Applicative f => Monad f where
  return :: a -> f a
  join f (f a) -> f a
  fmap f (a -> b) -> (f a -> f b) -- f = (a -> m b)

-- definition in GHC
class Applicative m => Monad m where
  return :: a -> m a
  (>>=)::m a -> (a -> m b) -> m b

-- f = (a -> m b)
m >>= f = join $ fmap (a -> m b) m b
m >>= f = join $ fmap f $ m b
m >>= f = join $ m (f b)
m >>= f = join $ m (m b)




\end{verbatim}
\item Maybe is Monad
\begin{verbatim}
instance Monad Maybe where
  return Nothing = Nothing
  (>>=) (Just a) f = Just f a

  addMaybe::Maybe Int -> Maybe Int -> Maybe Int
  addMaybe Nothing _ = Nothing
  addMaybe _ Nothing = Nothing
  addMaybe (Just a) (Just b) = Just (a + b)

  -- other implementation
  addMaybe::Maybe Int -> Maybe Int -> Maybe Int
  addMaybe m1 m2 = do
	  a <- m1
	  b <- m2
	  return (a + b)
\end{verbatim}
\end{enumerate}

\subsection{Applicative}
\label{sec:org5de62bf}
\begin{enumerate}
\item How to use Applicative
\item What is Applicative
\item What is the difference between Monad and Applicative
\end{enumerate}
\begin{verbatim}
class Functor f => Applicative f where
  pure:: a -> f a
 (<*>):: f (a -> b) -> f a -> f b

class Applicative f => Monad f where
  return:: a -> f a
  (>>=)::m a -> (a -> m b) -> m b

\end{verbatim}
\end{document}